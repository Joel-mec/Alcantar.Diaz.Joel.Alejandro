\documentclass[letterpaper]{article}
\usepackage{underscore}
\usepackage[left=2.0cm, right=2.0cm, top=2.0cm]{geometry}
\usepackage[utf8]{inputenc}
\usepackage{graphicx}
\usepackage{graphics}
\usepackage[spanish]{babel}
\usepackage{lipsum}
\usepackage{float}
\usepackage{subfigure}
\usepackage{biblatex}
\usepackage{csquotes}
\usepackage{color}

\title{EV\_2\_8\_calcular\_los\_parametros\_de\_circuitos\_de\_activación\_de\_transistores\_de\_potencia}
\author{Alcantar Diaz Joel Alejandro}
\date{29/Octubre/2019}

\begin{document}

\maketitle

\begin{center}
\vspace{2cm}
\includegraphics[scale=0.5]{UPZMGlog.png}\\
\vspace{2cm}
\begin{large}
    Universidad Politecnica de la Zona Metropolitana de Guadalajara.\\
    $4^{to}$ $"A"$
\end{large}
\end{center}

\newpage
\section{Transistor en corte y saturación.}
\begin{large}
    Se le llama corte y saturación cuando el es transistor se utiliza de tal forma que se comporta como un switch ya que al brindar la corriente necesaria por la base entra en saturacion y permite el paso de la corriente a traves de colector y emisor. Al privarle de esta corriente en la base entra en corte y no permite el paso de la corriente a traves del transistor.\\
\end{large}
\section{Calculo de resistencia para la base.}
\begin{large}
    Para calcular la resistencia se utiliza la siguiente formula:\\
    \begin{center}
        \begin{LARGE}
            $R=\frac{((V_e-V_{act})hFe)}{I_c}$
        \end{LARGE}
    \end{center}
    Donde R es la resistencia que se va a calcular, $V_e$ es el voltaje de entrada al transistor por la base, $V_{act}$ es el voltaje de saturacion del trancistor que puede ser 0.7 en los de silicio y 0.3 en los de germanio, hFe es el valor de amplificacion del trancistor que se puede encontrar con el Data Sheet o con el multimetro e $I_c$ es la corriente consumida por el circuito conectado al transistor.
\end{large}
\section{Ejemplo de uso.}
\begin{large}
    Supongamos que tenemos un TIP32C y queremos calcular la resistencia que se necesita para un circuito concetado a el que consume 300mA y un microcontrolador que manda un pulso de 5v, la formula se aplicaria de la siguiente manera:\\\\
    $R=\frac{((5V-0.7V)10)}{0.3A}$\\\\
    $R=\frac{((4.3)10)}{0.3A}$\\\\
    $R=\frac{43}{0.3A}$\\\\
    $R=143.3$\\\\
    Da como resultado una resistencia de 143.3$\Omega$, como no existen resistencia de ese valor de manera comercial se busca la mas cercana que es de 150$\Omega$.\\
    Entonces la resistencia mas ideal para este circuito es de 150$\Omega$, esta resistencia evita que el trancistor amplifique.
\end{large}
\begin{thebibliography}{}
\bibitem{formula}\textsc{Anonimo.} \textit{Calcular la resistencia para un transistor accionado por un microcontrolador.} Recuperdo el 29/10/2019 de:\\
https://www.sistemasorp.es/2011/10/05/calcular-la-resistencia-para-un-transistor-accionado-por-un-microcontrolador/\\
\bibitem{DS}\textsc{STMICROELECTRONICS.} \textit{TIP31A/31C TIP32A/32B/32C COMPLEMENTARY SILICON POWER TRANSISTORS.} Recuperdo el 29/10/2019 de:\\
https://pdf1.alldatasheet.com/datasheet-pdf/view/25376/STMICROELECTRONICS/TIP32C.html\\
\end{thebibliography}
\end{document}
